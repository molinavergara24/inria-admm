\documentclass[a4paper,11pt]{article}
%\documentclass[a4paper,11pt]{scrartcl}
%\AtBeginDvi{\special{pdf:mapfile dlbase14.map}}
%\AtBeginDvi{\special{pdf:mapfile ipaex.map}}
\usepackage{amsmath,amsthm,amssymb}
%\usepackage{enumitem}
\usepackage{algpseudocode,algorithm}
%\usepackage[dvipdfmx]{graphicx}
%\graphicspath{{./fig/}}
\usepackage[square,sort&compress,numbers]{natbib}
\usepackage{type1cm}
%\usepackage{url}
\usepackage{mathtools}\mathtoolsset{showonlyrefs=false}
%\usepackage{algpseudocode,algorithm}
%\usepackage{booktabs}
%
%\usepackage{subcaption}
%\captionsetup[subfigure]{labelformat=simple,font=small,
%  textfont=normalfont,singlelinecheck=off,
%  justification=raggedright}
%\captionsetup{subrefformat=simple}
%\renewcommand\thesubfigure{(\alph{subfigure})}

%\usepackage{helvet,courier}
%\usepackage{newtxtext,newtxmath}
%
%\usepackage{showkeys}
%
%%%%%%%%% \memo, \OMIT, \OMITREF, \titleetc %%%%%%%%%
\newcommand{\memo}[1]{\textbf{[MEMO:} #1 \ \textbf{ : end memo] }}  %%% For work
%% \renewcommand{\memo}[1]{}           %%% For FINAL
\newcommand{\OMIT}[1]{\textbf{[OMIT:} #1 \ \textbf{ --- end OMIT] }}  %%% For work
%%   \renewcommand{\OMIT}[1]{}            %%% For FINAL
%
\newcommand{\Authornote}{\renewcommand{\thefootnote}{\fnsymbol{footnote}}}
\newcommand{\authornote}{\Authornote\footnote}
%
\pagestyle{plain}

\setlength{\topmargin}{-14mm}
\setlength{\oddsidemargin}{-2mm}
\setlength{\textwidth}{166mm}
\setlength{\textheight}{50\baselineskip}
\addtolength{\textheight}{\topskip}
\renewcommand{\baselinestretch}{1.2}
\setlength{\parskip}{0.0pt}
%
\renewcommand{\topfraction}{0.9}
\renewcommand{\bottomfraction}{0.9}
\renewcommand{\dbltopfraction}{0.9}
\renewcommand{\textfraction}{0.1}
\renewcommand{\floatpagefraction}{0.9}
\renewcommand{\dblfloatpagefraction}{0.9}
\setcounter{topnumber}{3}
\setcounter{bottomnumber}{3}
\setcounter{totalnumber}{3}

%%%%%%%%%%%%%%%%%%%%%%%%%%%%%%%%%%%%%%%%%%%%%%%%

\theoremstyle{plain}
\newtheorem{theorem}{Theorem}[section]
\newtheorem{lemma}[theorem]{Lemma}
\newtheorem{assumption}[theorem]{Assumption}
%
\theoremstyle{definition}
\newtheorem{definition}[theorem]{Definition}
\newtheorem{proposition}[theorem]{����}
\newtheorem{example}[theorem]{��}
\newtheorem{corollary}[theorem]{�n}
%\newtheorem{algorithm}[theorem]{Algorithm}
%
\theoremstyle{remark}
\newtheorem{remark}[theorem]{Remark}
%
\newcommand{\refalg}[1]{Algorithm~\ref{#1}}
\newcommand{\refrem}[1]{Remark~\ref{#1}}
\newcommand{\reffig}[1]{Figure~\ref{#1}}
\newcommand{\reftab}[1]{Table~\ref{#1}}

\newcommand{\finbox}{\nolinebreak\hfill{\small $\blacksquare$}}

\newcommand{\MIN}{\mathop{\mathrm{Minimize}}}
\newcommand{\MAX}{\mathop{\mathrm{Maximize}}}
\newcommand{\Min}{\mathop{\mathrm{minimize}}}
\newcommand{\st}{\mathop{\mathrm{s.{\,}t.}}}
\newcommand{\ST}{\mathop{\mathrm{subject~to}}}
\newcommand{\sign}{\mathop{\mathrm{sgn}}\nolimits}
\newcommand{\domain}{\mathop{\mathrm{dom}}\nolimits}
\newcommand{\rank}{\mathop{\mathrm{rank}}\nolimits}
\newcommand{\boundary}{\mathop{\mathrm{bd}}\nolimits}
\newcommand{\diag}{\mathop{\mathrm{diag}}\nolimits}
\newcommand{\tr}{\mathop{\mathrm{tr}}\nolimits}
\newcommand{\prox}{\mathop{\boldsymbol{\mathrm{prox}}}\nolimits}
\newcommand{\argmin}{\operatornamewithlimits{\mathrm{arg\,min}}}
\newcommand{\argmax}{\operatornamewithlimits{\mathrm{arg\,max}}}

\newcommand{\overRe}{\ensuremath{\Re\cup\{+\infty\}}}

\renewcommand{\Re}{\ensuremath{\mathbb{R}}}

\newcommand{\bi}[1]{\ensuremath{\boldsymbol{#1}}}
\newcommand{\rr}[1]{\ensuremath{\mathrm{#1}}}
\newcommand{\bs}[1]{\ensuremath{\boldsymbol{\mathsf{#1}}}}

%%%%%%%%%%%%% Math Cal %%%%%%%%%%%%
%
\newcommand{\LC}{\ensuremath{\mathcal{L}}}

\usepackage{color}
\newcommand{\commentva}[1]{\textcolor{blue}{#1}}



\begin{document}

\noindent {Research Memorandum}

\hfill{\today}

\bigskip\bigskip%
\begin{center}
  {\Large\bfseries\sffamily% 
  Accelerated Alternating Direction Method of Multipliers }
  \par\medskip
  {\Large\bfseries\sffamily% 
  for Frictional Contact 
  \authornote[1]{%
  {\ttfamily /doc/contact/admm/}}%
  }%
  \par%
  \bigskip%
  {
  Yoshihiro Kanno~\authornote[2]{%
%  Corresponding author. 
  Laboratory for Future Interdisciplinary Research of Science and Technology, 
  Institute of Innovative Research, 
  Tokyo Institute of Technology, 
  Nagatsuta 4259, Yokohama 226-8503, Japan.
  E-mail: \texttt{kanno.y.af@m.titech.ac.jp}. 
%  Phone: +81-45-924-5364. 
%  Fax: +81-45-924-5977.
  }
  }
\end{center}


\section{Algorithm Coulomb}

The previous algorithm can be reformulated as well.
As the upshot, the fast ADMM (\refalg{alg:fast.ADMM.prototype}) 
for solving problem 
\eqref{P.quadratic.SOCP.2} is formally stated in \refalg{alg:fast.ADMM}. 

\begin{algorithm}
  \caption{Fast ADMM for problem \eqref{P.quadratic.SOCP.2}.}
  \label{alg:fast.ADMM}
  \begin{algorithmic}[1]
    \Require
    $\tilde{\bi{u}}^{0} = \hat{\bi{u}}^{0}$, $\bi{\zeta}^{0}=\hat{\bi{\zeta}}^{0}$, 
    $\tau_{0}=1$, and $\rho > 0$. 
    \For{$k=0,1,2,\dots$}
    \State \label{alg:fast.ADMM.linear}
    $\bi{v}^{k+1}$ solves 
    %\begin{align*}
      $\displaystyle
      \Bigl[ M + 
  \rho H^{\top} H \Bigr] \bi{v} 
  = -\bi{f} 
  + \rho H^{\top} (\tilde{\bi{u}}^k - \bi{b}(s) - \hat{\bi{\zeta}}^k)$   
    %\end{align*}

    \State \label{alg:fast.ADMM.projection}
    $\tilde{\bi{u}}^{k+1} := \Pi_{K_{e,\mu}^{*}}(H \bi{v}^{k+1}  + \hat{\bi{\zeta}}^k + \bi{b(s)})$ 
    \State
    $\bi{\zeta}^{k+1} := \hat{\bi{\zeta}}^k  + 
  H \bi{v}^{k+1} - \tilde{\bi{u}}^{k+1} + \bi{b}(s)$     
    \State
    $\displaystyle 
    \tau_{k+1} := \frac{1}{2} \Bigl( 1 + \sqrt{1 + 4\tau_{k}^{2}} \Bigr)$
    \State
    $\displaystyle
    \hat{\bi{u}}^{k+1}
    := \hat{\bi{u}}^{k} + \frac{\tau_{k}-1}{\tau_{k+1}} (\tilde{\bi{u}}^{k+1}-\tilde{\bi{u}}^{k})$ 
    \State
    $\displaystyle
    \hat{\bi{\zeta}}^{k+1}
    := \hat{\bi{\zeta}}^{k} + \frac{\tau_{k}-1}{\tau_{k+1}} (\bi{\zeta}^{k+1}-\bi{\zeta}^{k})$ 
    \EndFor
  \end{algorithmic}
\end{algorithm}


In step~\ref{alg:fast.ADMM.linear} of \refalg{alg:fast.ADMM}, 
the coefficient matrix of the system 
of linear equations is common to all the iterations. 
Hence, we carry out the Cholesky factorization only at the first 
iteration (i.e., $k=0$); at the following iterations, 
we can compute $\bi{v}^{k+1}$ only with the back-substitutions. 
The projection in step~\ref{alg:fast.ADMM.projection} can be computed 
explicitly by using the formula in \eqref{eq.projection.formula}. 
The computations in the other steps are only matrix-vector products and 
vector additions. 
Therefore, the computational cost required for one iteration is 
small, even for a large-scale problem. 

In a similar manner, we can apply 
\refalg{alg:fast.ADMM.restart.prototype} to problem 
\eqref{P.quadratic.SOCP.2}, as formally stated in \refalg{alg:fast.ADMM.restart}. 

\begin{algorithm}
  \caption{Fast ADMM for problem with restart \eqref{P.quadratic.SOCP.2}.}
  \label{alg:fast.ADMM.restart}
  \begin{algorithmic}[1]
    \Require
    $\tilde{\bi{u}}^{0} = \hat{\bi{u}}^{0}$, $\bi{\zeta}^{0}=\hat{\bi{\zeta}}^{0}$, 
    $\tau_{0}=1$, and $\rho > 0$. 
    \For{$k=0,1,2,\dots$}
    \State \label{alg:fast.ADMM.linear}
    $\bi{v}^{k+1}$ solves 
    %\begin{align*}
      $\displaystyle
      \Bigl[ M + 
  \rho H^{\top} H \Bigr] \bi{v} 
  = -\bi{f} 
  + \rho H^{\top} (\tilde{\bi{u}}^k - \bi{b}(s) - \hat{\bi{\zeta}}^k)$   
    %\end{align*}

    \State \label{alg:fast.ADMM.projection}
    $\tilde{\bi{u}}^{k+1} := \Pi_{K_{e,\mu}^{*}}(H \bi{v}^{k+1}  + \hat{\bi{\zeta}}^k + \bi{b(s)})$ 
    \State
    $\bi{\zeta}^{k+1} := \hat{\bi{\zeta}}^k  + 
  H \bi{v}^{k+1} - \tilde{\bi{u}}^{k+1} + \bi{b}(s)$  

    \If{$e_{k} < \eta e_{k-1}$}

    \State
    $\displaystyle 
    \tau_{k+1} := \frac{1}{2} \Bigl( 1 + \sqrt{1 + 4\tau_{k}^{2}} \Bigr)$

    \State
    $\displaystyle
    \hat{\bi{u}}^{k+1}
    := \hat{\bi{u}}^{k} + \frac{\tau_{k}-1}{\tau_{k+1}} (\tilde{\bi{u}}^{k+1}-\tilde{\bi{u}}^{k})$ 
    \State
    $\displaystyle
    \hat{\bi{\zeta}}^{k+1}
    := \hat{\bi{\zeta}}^{k} + \frac{\tau_{k}-1}{\tau_{k+1}} (\bi{\zeta}^{k+1}-\bi{\zeta}^{k})$ 
    
    \Else
    \State
    $\tau_{k+1} := 1$
    \State
    $\displaystyle
    \hat{\bi{u}}^{k+1}
    := \tilde{\bi{u}}^{k}$ 
    \State
    $\displaystyle
    \hat{\bi{\zeta}}^{k+1}
    := \bi{\zeta}^{k}$     
    \State
    $e_{k} \gets e_{k-1}/\eta$ 

    \EndIf

    \EndFor
  \end{algorithmic}
\end{algorithm}

\begin{itemize}
  \item To solve the friction problem, parameter $\bi{s}$ in problem 
	\eqref{P.original.friction.SOCP.1} should be updated. 
	This corresponds to updating $\bi{b}_{i}$ and $d_{i}$ in problem 
	\eqref{P.quadratic.SOCP.2}. 
	One obvious way is that, once we solve problem 
	\eqref{P.quadratic.SOCP.2} with fixed $\bi{b}_{i}$ and $d_{i}$ 
	by the fast ADMM (with restart), then update $\bi{b}_{i}$ and 
	$d_{i}$ by using the obtained solution, and repeat this 
	procedure. 
	Another possibility is to update $\bi{b}_{i}$ and $d_{i}$ at 
	each iteration of the fast ADMM (with restart). 
	Intuitively, the latter saves the total computational cost, but 
	stability of the algorithm in this case is not clear. 
\end{itemize}


\section{Projection onto second order cone}
\begin{align}
  \Pi_{K_{e,\mu}}(\bi{x}) = 
  \begin{dcases*}
	0
    & if $\|\bi{x_2}\| \leq  - \frac{1}{\mu} x_{1}$, \\
    \bi{x}
    & if $\|\bi{x_2}\| \leq \mu x_{1}$, \\
    \frac{1}{1+\mu^{2}} \left(x_{1} + \mu \| \bi{x}_{2} \|\right)
    \begin{bmatrix}
      1 \\    \mu \bi{x}_{2} / \| \bi{x}_{2} \| 
    \end{bmatrix}
    & if $- \mu \|\bi{x_2}\| < x_{1} < \frac{1}{\mu}\|\bi{x_2}\|$, \\
  \end{dcases*}  
\end{align}
\begin{align}
  \Pi_{K_{e,\mu}^{*}}(\bi{x}) = 
  \begin{dcases*}
	0
    & if $\|\bi{x_2}\| \leq  - \mu x_{1}$, \\
    \bi{x}
    & if $\|\bi{x_2}\| \leq \frac{1}{\mu} x_{1}$, \\
    \frac{\mu^{2}}{1+\mu^{2}} \left(x_{1} + \frac{1}{\mu} \| \bi{x}_{2} \|\right)
    \begin{bmatrix}
      1 \\    \frac{1}{\mu} \bi{x}_{2} / \| \bi{x}_{2} \| 
    \end{bmatrix}
    & if $- \frac{1}{\mu} \|\bi{x_2}\| < x_{1} < \mu \|\bi{x_2}\|$, \\
  \end{dcases*}  
\end{align}

\subsection{Projection onto Coulomb's friction cone}

For vector $\bi{x} = (x_{1},\bi{x}_{2}) \in \Re \times \Re^{n-1}$, its spectral 
factorization with respect to $K_{e,\mu}$ is defined by \cite{HYF05}
\begin{align}
  \bi{x} = \lambda_{1} \bi{u}^{1} + \lambda_{2} \bi{u}^{2} . 
\end{align}
Here, $\lambda_{1}$, $\lambda_{2} \in \Re$ are the spectral values given by 
\begin{align}
  \lambda_{i} = x_{1} + \left({-}1\right)^{i} \mu^{\left({-}1\right)^{i}} \| \bi{x}_{2} \| , 
\end{align}
and $\bi{u}^{1}$, $\bi{u}^{2}  \in \Re^{n}$ are the spectral vectors 
given by 
\begin{align}
  \bi{u}^{i} = 
  \begin{dcases*}
    \frac{1}{1+\mu^{2}}
    \begin{bmatrix}
      \mu^{2 \left(2-i\right)}  \\   \left({-}1\right)^{i} \mu \bi{x}_{2} / \| \bi{x}_{2} \| 
    \end{bmatrix}
    & if $\bi{x}_{2} \not= \bi{0}$, \\
    \frac{1}{1+\mu^{2}}
    \begin{bmatrix}
      \mu^{2 \left(2-i\right)}  \\   \left({-}1\right)^{i} \mu \bi{\omega}
    \end{bmatrix}
    & if $\bi{x}_{2} = \bi{0}$, \\
  \end{dcases*}  
\end{align}
with $\bi{\omega} \in \Re^{n-1}$ satisfying $\| \bi{\omega} \| = 1$. \\
For $\bi{x} \in \Re^{n}$, let $\Pi_{K_{e,\mu}}(\bi{x}) \in \Re^{n}$ 
denote the projection of $\bi{x}$ onto $K_{e,\mu}$, i.e., 
\begin{align}
  \Pi_{K_{e,\mu}}(\bi{x}) 
  = \argmin \{  \| \bi{x}' - \bi{x} \| 
  \mid \bi{x}' \in K_{e,\mu} \} . 
\end{align}
This can be computed explicitly as \cite{FLT01}
\begin{align}
  \Pi_{K_{e,\mu}}(\bi{x}) 
  = \max \{ 0,\lambda_{1} \} \bi{u}^{1} 
  + \max \{ 0,\lambda_{2} \} \bi{u}^{2} .
  \label{eq.projection.formula}
\end{align}
Therefore the projection of $\bi{x}$ onto $K_{e,\mu}$ could be written as follows
\begin{align}
  \Pi_{K_{e,\mu}}(\bi{x}) = 
  \begin{dcases*}
	0
    & if $-\bi{x} \in K_{e,\mu}^{*} \rightarrow \lambda_{i} \leq 0$ \\
    \bi{x}
    & if $\bi{x} \in K_{e,\mu} \rightarrow \lambda_{i} \geq 0$ \\
    \frac{\left(x_{1} + \mu \| \bi{x}_{2} \|\right)}{1+\mu^{2}} 
    \begin{bmatrix}
      1 \\    \mu \bi{x}_{2} / \| \bi{x}_{2} \| 
    \end{bmatrix}
    & if $-\bi{x} \not\in K_{e,\mu}^{*} \wedge \bi{x} \not\in K_{e,\mu} \rightarrow \lambda_{1} < 0 \wedge \lambda_{2} > 0$ \\
  \end{dcases*}  
\end{align}
Now, it is easy to see that the dual of $K_{e,\mu}$ is also a second-order cone
\begin{align}
  \ K_{e,\mu}^{*} = K_{e,\frac{1}{\mu}} = \{ (x_{1},\bi{x}_{2}) \in \Re \times \Re^{n-1}
  \mid \| \bi{x}_{2} \| \le \frac{1}{\mu} x_{1} \}
\end{align}
Consequently, the projection of $\bi{x}$ onto $K_{e,\mu}^{*}$ could be written as follows
\begin{align}
  \Pi_{K_{e,\mu}^{*}}(\bi{x}) = 
  \begin{dcases*}
	0
    & if $-\bi{x} \in K_{e,\mu} \rightarrow \lambda_{i}^{*} \leq 0$ \\
    \bi{x}
    & if $\bi{x} \in K_{e,\mu}^{*} \rightarrow \lambda_{i}^{*} \geq 0$ \\
    \frac{\mu^{2} \left(x_{1} + \frac{1}{\mu} \| \bi{x}_{2} \|\right)}{1+\mu^{2}} 
    \begin{bmatrix}
      1 \\    \frac{1}{\mu} \bi{x}_{2} / \| \bi{x}_{2} \| 
    \end{bmatrix}
    & if $-\bi{x} \not\in K_{e,\mu} \wedge \bi{x} \not\in K_{e,\mu}^{*} \rightarrow \lambda_{1}^{*} < 0 \wedge \lambda_{2}^{*} > 0$, \\
  \end{dcases*}  
\end{align}
where $\lambda_{i}^{*} = x_{1} + \left({-}1\right)^{i} \mu^{\left({-}1\right)^{i+1}} \| \bi{x}_{2} \|$.

\begin{thebibliography}{99}
\bibitem[\protect\citeauthoryear{Acary {\em et al.\/}}{2011}]{ACLM11}
  {V.~Acary, F.~Cadoux, C.~Lemar\'{e}chal, J.~Malick}:
  {A formulation of the linear discrete Coulomb friction problem 
    via convex optimization}.
  {\em ZAMM}, \textbf{91}, 155--175 (2011).

\bibitem[\protect\citeauthoryear{Acary and Cadoux}{2013}]{AC13}
  {V.~Acary, F.~Cadoux}:
  {Applications of an existence result for the Coulomb friction problem}.
  In: G.~E.~Stavroulakis (ed.),
  {\em Recent Advances in Contact Mechanics},
  pp.~45--66, Springer-Verlag, Berlin (2013).

\bibitem[\protect\citeauthoryear{Boyd {\em et al.\/}}{2010}]{BPCPE10}
  {S.~Boyd, N.~Parikh, E.~Chu, B.~Peleato, J.~Eckstein}:
  {Distributed optimization and statistical learning via the 
    alternating direction method of multipliers}.
  {\em Foundations and Trends in Machine Learning},
  \textbf{3}, 1--122 (2010).

\bibitem[\protect\citeauthoryear{Fukushima {\em et al.\/}}{2001}]{FLT01}
  {M.~Fukushima, Z.-Q.~Luo, P.~Tseng}:
  {Smoothing functions for second-order-cone complementarity problems}.
  {\em SIAM Journal on optimization},
  \textbf{12}, 436--460 (2001).

\bibitem[\protect\citeauthoryear{Giselsson and Boyd}{2014}]{GB14}
  {P.~Giselsson, S.~Boyd}:
  {Diagonal scaling in Douglas--Rachford splitting and ADMM}.
  {\em The 53rd IEEE Conference on Decision and Control},
  5033--5039, Los Angeles (2014).

\bibitem[\protect\citeauthoryear{Goldstein {\em et al.\/}}{2014}]{GOdSB14}
  {T.~Goldstein, B.~O'Donoghue, S.~Setzer, R.~Baraniuk}:
  {Fast alternating direction optimization methods}.
  {\em SIAM Journal on Imaging Sciences},
  \textbf{7}, 1588--1623 (2014).

\bibitem[\protect\citeauthoryear{Hager {\em et al.\/}}{2015}]{HNYZ15}
  {W.~Hager, C.~Ngo, M.~Yashtini, H.-C.~Zhang}:
  {An alternating direction approximate Newton algorithm 
    for ill-conditioned inverse problems with application to parallel MRI}.
  {\em Journal of the Operations Research Society of China},
  \textbf{3}, 139--162 (2015).

\bibitem[\protect\citeauthoryear{Hayashi {\em et al.\/}}{2005}]{HYF05}
  {S.~Hayashi, N.~Yamashita, M.~Fukushima}:
  {A combined smoothing and regularization method for 
    monotone second-order cone complementarity problems}.
  {\em SIAM Journal on Optimization},
  \textbf{15}, 593-615 (2005).

\bibitem[\protect\citeauthoryear{Martins and Pinto da Costa}{2000}]{MPC00}
  {J.~M.~C.~Martins, A.~Pinto da Costa}:
  {Stability of finite-dimensional nonlinear elastic systems with 
    unilateral contact and friction}.
  {\em International Journal of Solids and Structures},
  \textbf{37}, 2519--2564 (2000).

\bibitem[\protect\citeauthoryear{Martins {\em et al.\/}}{2002}]{MPCS02}
  {J.~M.~C.~Martins, A.~Pinto da Costa, F.~M.~F.~Sim\~{o}es}:
  {Some notes on friction and instabilities}.
  In: J.~A.~C.~Martins, M.~Raous (eds.),
  {\em Friction and Instabilities},
  pp.~65--136, Springer-Verlag, Wien (2002).

\bibitem[\protect\citeauthoryear{Pinto da Costa {\em et al.\/}}{2004}]{PCMFJ04}
  {A.~Pinto da Costa, J.~A.~C.~Martins, I.~N.~Figueiredo, 
    J.~J.~J\'{u}dice}:
  {The directional instability problem in systems with 
    frictional contacts}.
  {\em Computer Methods in Applied Mechanics and Engineering},
  \textbf{193}, 357--384 (2004).

%\bibitem[\protect\citeauthoryear{Tseng}{1991}]{Tse91}
%  {P.~Tseng}:
%  {Applications of a splitting algorithm to decomposition 
%    in convex programming and variational inequalities}.
%  {\em SIAM Journal on Control and Optimization},
%  \textbf{29}, 119--138 (1991).

\end{thebibliography}



\clearpage

\tableofcontents

\end{document}

%%% Local Variables:
%%% mode: latex
%%% TeX-master: t
%%% End:
\grid
